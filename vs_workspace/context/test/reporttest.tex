\documentclass[a4paper,10pt,twocolumn]{jsarticle}
\pagestyle{empty}	
\usepackage[top=15truemm,bottom=15truemm,left=25truemm,right=25truemm]{geometry}
% 数式
\usepackage{amsmath,amsfonts}
\usepackage{bm}
\usepackage{tikz}
% 画像
\usepackage{graphicx}


\usepackage{otf}

\begin{document}
\title{\Large {\textbf{模型自動車を用いた遠隔型実車運転システム}}}
\author{\textbf{E1621\CID{8705}橋 尚太郎}\and \textbf{指導教員 }\textbf{野村 隼人}}
\date{}
\maketitle
\section{はじめに}
自動車を遠隔運転すれば,操縦者が実車に搭乗しなくても,人や荷物などを遠方に運ぶことができる.
近年,自動車の遠隔運転システムの操作性に関する研究がすすめられ,自動運転AIの省略化,製作コスト削減に期待されている.
遠隔運転では,カメラ映像が送信される遠隔モニタを操縦者が確認しながら操作するため,カメラ映像の追従精度が操作に影響を与えるとされている.
また,遠隔運転には自動車免許が必要であるが,対象をラジコンカー(RCカー)とすることで,免許の有無に関わらず,
遠隔運転の練習が行える.本研究ではその点を目的として,遠隔運転システムとラジコン用操作機器による操縦を,視点や感度に基づいて比較評価する.
その結果,映像遅延による影響が少ない低速域での遠隔運転を快適に行うことが出来れば,模型自動車の体感速度の大きさを利用し,高速域での実車の遠隔運転を再現した実験等に応用することができると考えられる.
\section{本研究の概要}
図1のように,操作者が対象側のカメラ映像を確認して,ハンドル・フットペダル型ゲームコントローラの入力検出で模型自動車を遠隔運転するシステムは,前進・後退・旋回操作の遠隔運転練習を用途とした.
また,操作者の運転経験の有無や,通常のRCカー操作用コントローラ(プロポ)との比較で,遠隔操作性に与える影響を評価する.
\section{遠隔操作性の比較について}
評価に用いるプロポには,一般的に以下のような特徴がある.
\begin{itemize}
    \item 操作方法が実車と異なる.
    \item RCカーと操作目標の一関係を把握して操作する必要がある.
    \item 操作入力に対する応答が敏感である.
\end{itemize}
\section{製作したシステムについて}
\subsection{模型自動車}
遠隔操作を行う模型自動車に用いた,タミヤ1/10XBシリーズは,モータの制御信号を受信するための端子が設けられている.
Arduinoの汎用入出力端子に接続することで,模型自動車を操作することができる.
\subsection{カメラ}
カメラは,周囲の状況を広範囲に確認しやすくした一人称(実車操縦席)と三人称(一般的なラジコン)視点を操作性比較に加味するために用いた.
360度カメラのGoProMAXを,カメラ映像のリアルタイム送信,映像再生するため,GoPro API for Pythonによって使用した.
\subsection{コントローラ}
操作者の入力検出に使用したLogicool社製のGT Force Proにはハンドル部の回転おtペダル部の踏み込みを検知するセンサが搭載されていて,その出力を
Processing言語のGameControlPlusで読み取った.また,遠隔操作の無線通信はXbeeモジュールによるシリアル通信機能で行う.
\section{評価実験}
評価実験の目的は,異なる操作機器と視点における遠隔操作を比較し,遠隔操作に際し,運転免許・経験の有無を踏まえ,遠隔運転とプロポ操作の比較によって,
遠隔運転の操作性を庁さすることである.3人の冷え検車が各システムにより,図2に示すコース1周(約51.2 m)を遠隔操作する際の周回時間を比較する.
 操作視点を図1のカメラ映像と図3の三人称(RCカーを直接目で追う)とし,被験者3名に遠隔操作感覚についてアンケート調査を行った.
\subsection{実験結果}
遠隔運転システムとRC操作用プロポで遠隔操作をお粉ttあ際のコース集会時間[秒]と,アンケート結果を評価した.周回時間は,3周走行して平均を取った値とした.
表1,2に示す.表1を確認すると,カメラの有無で,ハンドル操作では周回時間の差が最大で10秒以内に対し,プロポの場合,平均して10秒以上の差が生じることが分かる.
また,プロポでは最高速度が30~40 km/hであり,遠隔運転システムの最高速度が8 km/h程度でありながら,操作環境の違いによる周回時間に極端な差異がない点が,今回の実験により確認された.
表2は,被験者の数を考慮すると多数の意見ではないが,3人中全員が遠隔運転システムの操作性を快適としている点は,仮定通りであった.
\section{今後について}
本実験中,遠隔運転システムにおいて,ハンドルとペダルの同時入力で車体が静止する以上が発生した.しかし,操作システムの差異がない事が今回確認されたため,以上を改善nし,今後同様の実験を行うことで周回時間の短縮が期待される.
加えて,両システムの操作性評価の精度を高めるため,被験者数を拡大する予定である.
\end{document}