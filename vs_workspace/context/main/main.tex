\documentclass[dvipdfmx,a4paper,10pt]{jsarticle}% ドライバ dvipdfmx を指定する
\usepackage{tikz}
\begin{document}
\usetikzlibrary {arrows.meta,automata,positioning,shadows}
\begin{tikzpicture}[->,>={Stealth[round]},shorten >=1pt,auto,node distance=2.8cm,on grid,semithick,
every state/.style={fill=red,draw=none,circular drop shadow,text=white}]
\node[initial,state] (A) {$q_a$};
\node[state] (B) [above right=of A] {$q_b$};
\node[state] (D) [below right=of A] {$q_d$};
\node[state] (C) [below right=of B] {$q_c$};
\node[state] (E) [below=of D] {$q_e$};
\path (A) edge node {0,1,L} (B)
edge node {1,1,R} (C)
(B) edge [loop above] node {1,1,L} (B)
edge node {0,1,L} (C)
(C) edge node {0,1,L} (D)
edge [bend left] node {1,0,R} (E)
(D) edge [loop below] node {1,1,R} (D)
edge node {0,1,R} (A)
(E) edge [bend left] node {1,0,R} (A);
\node [right=1cm,text width=8cm] at (C)
{
The current candidate for the busy beaver for five states. It is
presumed that this Turing machine writes a maximum number of
$1$'s before halting among all Turing machines with five states
and the tape alphabet $\{0, 1\}$. Proving this conjecture is an
open research problem.hog
};
\end{tikzpicture}
\end{document}